\chapter{Bevezetés}

Vannak felhasználók akik szeretnének \textit{SQL}-t tanulni, ennek több módja is van.
Egyik mód egy integrált fejlesztői környezet használata, ami kényelmes
felületet biztosít a tanuláshoz, többek között a szintaxis kiemelésnek,
illetve a könnyen szerkeszthető, és futtatható lekérdezések segítségével.
A szerkesztőfelületnek köszönhetően ha hibát ejt a felhasználó azt könnyen
javíthatja, és elmentheti munkáját, illetve betölthet korábban létrehozott
\textit{SQL} fájlokat is. A szintaxis kiemelésnek köszönhetően így könnyű észrevenni,
és javítani a hibákat, valamint lényegesen gyorsabban meg lehet tanulni az \textit{SQL}
nyelvet. Bizonyos műveleteket akár az \textit{SQL} teljes ismerete nélkül is elvégezhet bárki:
adatbázis objektumok megtekintését, törlését.
Továbbá tapasztalt felhasználóknak is hasznos eszköz, könnyebben tudnak \textit{SQL}
lekérdezéseket, illetve szkripteket készíteni és futtatni. Emellett képes megjeleníteni
SQL lekérdezéseknek a lekérdezési tervét.

Ez a program egy ilyen integrált fejlesztői környezetet valósít meg. Ezen felül a lekérdezések
diagramokba rendezhetőek, így egy szemléletesebb eszközt adva adataink megjelenítésére. Mivel az
implementáció Qt segítségével történt, így könnyen bővíthető a program további adatbázis szerverek
használatára, de a program csak Oracle SQL adatbázis szerverhez képes csatlakozni.
A programot \textit{DiagramQuery} néven fogjuk a továbbiakban említeni.