\chapter{Bevezetés}

A szoftverfejlesztést nagyban meghatározza milyen fejlesztői környezetben végezzük a fejlesztést. A fejlesztői
környezet az, amiben létrejön a szoftverfejlesztés folyamata során a kész program, avagy amiben
lehet fordítani, és tesztelni az általunk írt kódot. Egy integrált fejlesztői környezet ezt bővíti
ki olyan eszközökkel, amelyek segítségével a fejlesztés kényelmesebb, gyorsabb lesz, így egy
hatékonyabb eszközt adva a fejlesztő kezébe. Ezzel szemben nem csak tapasztalt felhasználók tarthatnak igényt
egy ilyen környezetre, hanem kevésbé tapasztalt, jelenleg tanuló felhasználók is.

A program ezeknek a felhasználóknak is hasznos lehet, akik jelenleg csak ismerkednek az SQL-el. 
Ezen felhasználók többféleképpen is nekikezdhetnek a tanulásnak, erre az
egyik mód egy integrált fejlesztői környezet használata, ami kényelmes
felületet biztosít a tanuláshoz, többek között a szintaxis kiemelésnek,
illetve a könnyen szerkeszthető, és futtatható lekérdezések segítségével.

Felmerülhet a kérdés, hogy pontosan mire is jó az SQL?
Az SQL, azon belül is az Oracle SQL nyelv mögötti szerver képes biztonságosan tárolni nagy
mennyiségű adatokat, hiba esetén helyreállítani azokat, illetve lehetőség van a különböző adattáblák indexelésére
ami felgyorsíthatja a lekérdezéseket, de ezen kívül számos pozitív (és persze negatív) tulajdonsága is van,
ami jelenlegi program szempontjából nem annyira lényeges. Felmerülhet az igény, hogy ezeket az adatmennyiségeket
(például napi adatgyűjtés a felhasználók szokásairól) valamilyen szempont alapján megfigyeljük, megjelenítsük.
Erre szolgál az ebben a programban létrehozható diagramok: ha megfelelően lekérdezzük az adatokat jól látható
eredményeket kapunk, amik segíthetnek bizonyos szolgáltatások fejlesztésében.

A szerkesztőfelületnek köszönhetően ha hiba történik az könnyen
javítható, valamint elmentheti munkáját, betölthet korábban létrehozott
SQL fájlokat is. A szintaxis kiemelésnek köszönhetően könnyű észrevenni,
és javítani a hibákat, sőt lényegesen gyorsabban meg lehet tanulni az SQL
nyelvet. Bizonyos műveleteket akár az SQL teljes ismerete nélkül is elvégezhet bárki:
adatbázis objektumok megtekintését, törlését, esetlegesen korábban létrehozott lekérdezésekből
diagramok megjelenítését.
Továbbá tapasztalt felhasználóknak is hasznos eszköz, könnyebben tudnak SQL
lekérdezéseket, illetve szkripteket készíteni, és futtatni. Emellett képes megjeleníteni
SQL lekérdezéseknek a lekérdezési tervét.

Mindezen funkciók megvalósítása volt a cél a program írása közben. Mivel az
implementáció Qt keretrendszer segítségével történt, így könnyen bővíthető a program további adatbázis szerverek
használatával, azonban a program csak Oracle SQL adatbázis szerverhez képes csatlakozni.
A programra \textit{DiagramQuery} néven fogok a továbbiakban hivatkozni.