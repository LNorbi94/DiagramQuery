\chapter{Felhasználói dokumentáció}
\textit{DiagramQuery} első sorban lehetőséget biztosít egy Oracle SQL adatbázishoz való kapcsolódásra.
A kapcsolatokat tartalmazó adatokat el is lehet menteni (illetve lehetőség van kézzel leírni,
ez kifejtésre kerül az első szekcióban.), illetve betölteni egy .xml formátumú fájlból.

A kapcsolat felépítése után lehetőség van egy szerkesztő ablakban SQL valamint PL/SQL parancsokat írni,
illetve diagramokat is itt lehet készíteni, ennek részletezése a 3. szekcióban történik meg.

\section{Telepítés és beállítás}
A \textit{DiagramQuery} telepítése történhet bármely operációs rendszeren amelyen rendelkezésre áll a
Clang 4.8 és a Qt 4.8-as verziója. A program ezeken lett tesztelve, természetesen működhet más verzióval is,
de nem garantált. Ezeken felül szükség van OCI 1.2-re is, és fordítani kell a megfelelő Qt-s plugint.
Illetve bizonyos linux disztribuciók alatt lehetőség van közvetlen letöltésre is.

Lehetőség van későbbi fejlesztések során további adatbázis-kezelők hozzáadására, ezekhez elég lesz a Clang, illetve Qt birtoklása.
Telepítés után lehetőség van manuálisan létrehozni kapcsolatokat. Ha ezt kívánja tenni hozzon létre egy
\textit{Connections} mappát a futtatható állomány mellé, és hozzon létre egy .xml kiterjesztésű fájlt a kívánt névvel.
A fájl felépítése a következő legyen:
\begin{itemize}
  \item Connection címkék között legyen elhelyezve az egész, illetve legyen a címkének egy attribútuma: name, a kapcsolat nevével,
  \item host címkék között a kapcsolat címe legyen,
  \item port címkék között legyen megadva a kapcsolat portja,
  \item service címkék között legyen a szervíz neve,
  \item illetve username címkék között a bejelentkezéshez szükséges felhasználónév.
\end{itemize}
Ha valamelyik mezőt nem szeretné eltárolni, akkor azt címkével együtt hagyja ki.
Egy konkrét példa a fájlra: (port megadása nélkül)

\begin{lstlisting}
  <?xml version="1.0" encoding="UTF-8"?>
  <Connection name="Elte">
      <host>aramis.inf.elte.hu</host>
      <service>eszakigrid97</service>
      <username>A8UZ7T</username>
  </Connection>
\end{lstlisting}

Miután létrehoztuk a fájlt el is kezdhetjük a program használatát, további beállításra nincs szükség.

\section{Kapcsolódás}
\begin{figure}[ht]
  \includegraphics{connect}
 \caption{Kapcsolódási ablak}
\end{figure}
A mentett kapcsolatok kezelésére itt is lehetőség van. A \textit{delete} billentyű lenyomására a kijelölt kapcsolat törlődik.
Lehetőség van az aktuális, képernyőn látható adatok mentésére is. Ekkor egy nevet meg kell adni a kapcsolatnak, ami nem lehet üres.
Létező név esetén lehetőség van a korábbi kapcsolat felülírására.
Az adatok helyes megadása után, illetve ha létrejön sikeresen a kapcsolat az adatbázissal, akkor a program beléptet a szerkesztő felülethez.
Hiba esetén kiírja a hiba kódot, amivel keresse meg adatbázisának adminisztrátorát, vagy nézzen utána az Oracle oldalán.

Néhány példa a hibakódokra:
\begin{itemize}
  \item \textbf{ORA-12560}: TNS:protocol adapter error; Unable to logon. Ezt okozhatja internet kapcsolat hiánya,
  vagy hiányzó szervernév, azaz ha nem lehet kiépíteni kapcsolatot a szerverrel bármilyen okból.
  \item \textbf{ORA-01005}: null password given; logon denied; Unable to logon. Ezt a hibaüzenetet akkor fogja ön kapni,
  ha a jelszó mezőt üresen hagyja. (Ha üres a jelszó mező akkor hibás felhasználónév esetén is ez a hibaüzenet jön elő)
  \item \textbf{ORA-01017}: invalid username/password; logon denied; Unable to logon. Hibás felhasználónév vagy jelszó esetén.
\end{itemize}

Ezek a hibakódok természetesen más esetben is előjöhetnek, ha biztos benne hogy minden rendben van az adatokkal, és a
kapcsolattal, akkor kérjen meg egy hozzáértőt a probléma elhárításában.

Az \href{https://docs.oracle.com/cd/B28359_01/server.111/b28278/toc.htm}{Oracle Docs} oldalon megtalálható minden hibakód, és annak oka (angolul).
Legtöbb esetben elég beszédesek, így meg lehet őket érteni a dokumentáció nélkül is.
A későbbiekben is ezen hibakódok vannak használva (és logolva), így ezt az oldalt érdemes lementeni.

\section{Grafikus felület}
\begin{figure}[ht]
    \includegraphics[width=1.0\textwidth]{gui}
 \caption{Felhasználói felület}
\end{figure}
Sikeres belépés esetén ez a képernyő fogadja a felhasználókat.
A következő billentyűparancsok használhatak az ablakon belül:
\begin{itemize}
  \item \textbf{CTRL+W, illetve CTRL+SHIFT+W}: bezárja alsó, illetve felső lapot. Az alapértelmezett lapokat nem lehet bezárni.
  \item \textbf{CTRL+ENTER}: A szerkesztőben lévő lekérdezést vagy scriptet végrehajtja. (Kurzor helyzetétől függően.)
  \item \textbf{F4}: Lekérdezési terv létrehozása.
  \item \textbf{F3}: Kijelölt lekérdezés végrehajtása.
  \item \textbf{DELETE}: Adatbázis objektum törlése a bal oldali fából.
\end{itemize}

A menüben lévő parancsok (Megnyitás, Mentés, Kilépés, stb...) maguktól értetődnek, egy dologra azonban
felhívnám a figyelmét: Ha időtúllépés miatt kilépteti az adatbázis (ezt az adatbázis karbantartójától megkérdezheti mennyi idő) akkor
érdemes az újrakapcsolódást választani, mivel ha lekérdezést próbál végrehajtani, akkor az adatbázis sokáig nem fog válaszolni.

\subsection{Lekérdezések}
A program szerkesztő felületében lehetőség van sql lekérdezések, illetve PL/SQL szkriptek végrehajtására. SQL* Plus szkriptek viszont
nem működnek. Ezen utasítások máshol ki vannak fejtve, így ezt itt nem tenném meg. A programban szintaxis kiemeléssel jelennek meg ezek
az utasítások, így tanulás közben is használható. (De automatikus kitöltés nincsen)

\subsection{Diagramok}
Kettő különböző típusú diagram lett implementálva a programban, a kör- és oszlopdiagram. A továbbiakban
ezeknek a szintaxisát részletezem, illetve leírom mikre használhatóak, mivel ezek más adatok
reprezentációjára alkalmasak.

Néhány elvárás minden diagramtípussal kapcsolatban:
\begin{itemize}
  \item Minden lekérdezésnek legalább 1 sort kell tartalmaznia.
  \item Típustól függően meg van határozva egy maximum számú sor is.
  \item GROUP BY típusú lekérdezés használható csak.
  \item Felesleges oszlopok nem szerepelhetnek a lekérdezésben. (azaz amiket nem használ a diagram)
\end{itemize}

\subsubsection{Kördiagram}
\begin{figure}[ht]
  \begin{center}
    \includegraphics[width=1.0\textwidth]{piechart}
  \end{center}
 \caption{Kördiagram}
\end{figure}

A kördiagram kategórikus adatok megjelenítésére használható.
Szintaxisa a következő:
\textbf{{\color{awesomeblue} MAKE CHART PIECHART }} ( \textbf{$<$SQL lekérdezés$>$} );

\textbf{Kördiagram tulajdonságai:}
\begin{itemize}
  \item Legalább 2, legfeljebb 20 elemmel használható.
  \item 2 oszlopos lekérdezéssel lehet használni; második oszlopnak egy aggregáló függvénynek kell lennie.
  \item Érdemes csökkenő sorrendben lekérni az elemeket, de ez teljes mértékben a felhasználóra van bízva.
  \item A címkékben százalékosan jelennek meg az adatok. (Kerekítés miatt lehet benne hiba)
\end{itemize}

\subsubsection{Oszlopdiagram}
\begin{figure}[ht]
  \begin{center}
    \includegraphics[width=1.0\textwidth]{barchart}
  \end{center}
 \caption{Oszlopdiagram}
\end{figure}

Az oszlopdiagram kategórikus adatok megjelenítésére használható.
Szintaxisa a következő:
\textbf{{\color{awesomeblue} MAKE CHART BARCHART }} ( \textbf{$<$SQL lekérdezés$>$} );

\textbf{Oszlopdiagram tulajdonságai:}
\begin{itemize}
  \item Legalább 2, legfeljebb 10 elemmel használható.
  \item 2 oszlopos lekérdezéssel lehet használni; második oszlopnak egy aggregáló függvénynek kell lennie.
  \item Érdemes eredeti sorrendben meghagyni az adatokat, hogy könnyen megfigyelhessük a kiugró adatokat.
\end{itemize}