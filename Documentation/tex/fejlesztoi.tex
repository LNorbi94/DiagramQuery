%A Fejlesztői dokumentáció tartalmazza
%- a probléma részletes specifikációját,
%- a felhasznált módszerek részletes leírását, a használt fogalmak definícióját,
%- a program logikai és fizikai szerkezetének leírását (adatszerkezetek, adatbázisok,
%modulfelbontás),
%- a tesztelési tervet és a tesztelés eredményeit.

\chapter{Fejlesztői dokumentáció}

Egy integrált fejlesztői környezetnél többféle igény is megfogalmazható. Valamelyik fejlesztőnek
olyanra van szüksége, amely ugyan kevesebbet tud, de gyorsabban indítható, esetlegesen gyorsabban
végezhető vele a munka. SQL-hez található több integrált fejlesztői környezet, ilyen az Oracle-nél
például az SQLDeveloper.
Az én programom az SQL Developerhez képest kevesebb komponents tartalmaz, így gyorsabb is. Tökéletes
eszköz ha valaki csak a lekérdezésekre próbál koncentrálni. Ennek ellenére rendelkezik olyan
funkcióval amivel az SQL Developer nem: lehet vele diagramokat készíteni.
A fejlesztéshez szükséges eszközök kiválasztásnál tehát szempont volt a hatékonyság, valamint hogy
ennek ellenére mégis nyújtson valami újat, ezért esett a választásom a Qt-ra, ami egy C++ keretrendszer.
Nem csak hatékony kódot lehet vele készíteni, szükség esetén magas absztrakció is használható, és nem
csak egy operáció rendszeren fordítható a Qt-ban készített programok.

\section{Tesztelés}