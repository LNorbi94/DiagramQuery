%A Fejlesztői dokumentáció tartalmazza
%- a probléma részletes specifikációját,
%- a felhasznált módszerek részletes leírását, a használt fogalmak definícióját,
%- a program logikai és fizikai szerkezetének leírását (adatszerkezetek, adatbázisok,
%modulfelbontás),
%- a tesztelési tervet és a tesztelés eredményeit.

\chapter{Fejlesztői dokumentáció}

Egy integrált fejlesztői környezetnél többféle igény is megfogalmazható. Valamelyik fejlesztőnek olyanra van szüksége,
amely ugyan kevesebbet tud, de gyorsabban indítható, esetlegesen gyorsabban végezhető vele a munka.
SQL-hez található több integrált fejlesztői környezet, ilyen az Oracle-nél például az SQLDeveloper.
Az én programomban

\section{Tesztelés}