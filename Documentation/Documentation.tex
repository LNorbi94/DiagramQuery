\documentclass{elteikthesis}

\usepackage{ucs}
\usepackage[utf8x]{inputenc}
\usepackage[T1]{fontenc}
\usepackage[english,hungarian]{babel}
\selectlanguage{hungarian}

\title{Integrált adatbázis-fejlesztői környezet SQL-hez}
\author{Lestár Norbert}
\supervisor{dr Nikovits Tibor}
\supervisorstitle{Mestertanár}
\period{programtervező informatikus BSc}
\thesisyear{2017}
\department{Információs Rendszerek Tanszék}

\begin{document}

\frontmatter

	\maketitle
	\tableofcontents
	
\mainmatter

\chapter{Bevezetés} 
Nem minden felhasználó ért az SQL-hez, részben ezért lehet szükség egy integrált fejlesztői
környezetre. A szintaxis kiemelésnek köszönhetően így könnyű észrevenni és javítani a hibákat,
valamint lényegesen gyorsabban meg lehet tanulni az SQL-t. Bizonyos műveleteket akár az SQL teljes
ismerete nélkül is elvégezhet bárki. Továbbá tapasztalt felhasználóknak is hasznos eszköz, mert
letisztult és átlátható felületet biztosít az adatbázis objektumainak megtekintésére. Ezen felül a
diagramok segítségével könnyen leolvashatóvá válnak az adatbázis különböző tulajdonságai, valamint
az egyedi felhasználó által létrehozott táblák is.
A megvalósított program képes ezekre: egy szerkesztőn lehetőség nyílik SQL, illetve PL/SQL kód
leírására, esetleg ha már van kész kód akkor annak betöltésére. Oracle SQL adatbázishoz lehet
csatlakozni, és csatlakozás után a felhasználói felület segítségével különböző dolgokra nyílik lehetőség:
\begin{itemize}
  \item Adatbázisban megtalálható objektumok megtekintésére, törlésére, illetve újak létrehozására,
  \item program használata során kipróbált SQL, illetve PL/SQL parancsok mentésére,
  \item különböző diagramok (kördiagram, oszlopdiagram) megtekintésére lekérdezések segítségével,
  \item valamint lekérdezési tervek megtekintésére is.
\end{itemize}


\chapter{Felhasználói dokumentáció}

\chapter{Fejlesztői dokumentáció}

\end{document}